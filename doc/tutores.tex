\section{Descrição do Problema (Tutores - URI 2120)} \label{sec:tutores}
O problema Tutores\footnote{\url{https://www.urionlinejudge.com.br/judge/en/problems/view/2120}}
consiste em dado uma ordem de inserções de matrículas (chaves) em uma Árvore
Binária de Busca (ABB), deve ser respondido, para várias matrículas, qual a
matrícula do nó pai do nó em que essa matrícula está, ou seja, seu tutor.

Formalmente, dado uma ordem de inserção de matrículas $O = \{X_1, X_2, \cdots, X_n\}$ e
uma ABB $T$ em que foram inseridas as matrículas $O$. Considere $p(X)$ como a matrícula
do nó pai do nó em que a matrícula $X$ está em $T$. Para uma sequência $I = \{i_1, i_2, \cdots, i_m\}$,
a solução deve apresentar uma sequência de resposta $Y = \{p(X_{i_1}), p(X_{i_2}), \cdots, p(X_{i_m})\}$.

Por exemplo, suponhamos a ordem $\hat{O} = \{2, 4, 5, 6, 1\}$ e a sequência $\hat{I} = \{2, 3, 4\}$.
A sequência de resposta é $Y = \{p(X_2 = 4) = 2, p(X_3 = 5) = 4, p(X_4 = 6) = 5\}$. Os valores
de $p$ podem ser visualizados na árvore binária de busca $T$ desse exemplo que 
está na figura \ref{fig:abb_exemplo}.

\begin{figure}[!htb]
\centering
\begin{forest}
for tree={
    grow=south,
    circle, draw, minimum size=3ex, inner sep=1pt,
    s sep=7mm
        }
[2
	[1]
	[4
		[5
			[6]
		]
	]
]
\end{forest}
\caption{Árvore Binária de Busca $T$ gerada a partir da ordem de inserção de chaves $\hat{O}$}
\label{fig:abb_exemplo}
\end{figure}