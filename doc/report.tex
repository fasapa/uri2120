\documentclass[12pt]{article}

\usepackage{sbc-template}
\usepackage{graphicx,url}
\usepackage{amsmath}
\usepackage{amssymb}

\usepackage{indentfirst}
\usepackage{listings}
\usepackage{algorithm}
\floatname{algorithm}{Algoritmo}
\usepackage{algorithmicx}
\usepackage{algpseudocode}
% \usepackage[portuguese]{algorithm2e}
\usepackage[T1]{fontenc}
\usepackage[utf8]{inputenc}  
\usepackage[brazil]{babel}

\usepackage{tikz}
\usetikzlibrary{shapes.multipart,positioning,arrows,calc}
\tikzset{
  listnode/.style={
    rectangle split,rectangle split parts=2,draw,rectangle split horizontal,fill=blue!20
  },
  startnode/.style={
    draw,minimum width=1.5cm,minimum height=.75cm
  }
}

\usepackage[edges]{forest} % draw trees

\sloppy

\title{Estruturas de Dados e Projetos de Algoritmos\\Tutores (URI 2120)}
\author{Fabrício S. Paranhos\inst{1} e Leandro A. Vianna\inst{1}}
\address{Instituto de Informática -- Universidade Federal de Goiás (UFG)}

\begin{document} 

\maketitle

% \begin{abstract}
%   This meta-paper describes the style to be used in articles and short papers
%   for SBC conferences. For papers in English, you should add just an abstract
%   while for the papers in Portuguese, we also ask for an abstract in
%   Portuguese (``resumo''). In both cases, abstracts should not have more than
%   10 lines and must be in the first page of the paper.
% \end{abstract}
     
\begin{resumo} 
  No desenvolvimento de uma solução para associar tutores a alunos, Lucca
  Siaudzionis fornece uma solução elegante: utilizar uma árvore binária de busca,
  aonde os alunos são representado por nós e seus tutores são seus respectivos
  pais na árvore. Entretanto, devido a problemas técnicos, reconstruir a mesma
  torna-se impraticável, com complexidade de inserção na ordem O($n$).
  Portanto, desenvolvemos um novo algoritmo, capaz de lidar com as demandas e limitações
  temporais do problema, com o auxílio de uma árvore binária balançeada
  auxiliar, com inserção de complexidade amortizada O($\lg n$). Permitindo a consulta aos
  tutores sem a necessidade de armazenamento da árvore binária de busca. 
\end{resumo}

\section{Descrição do Problema (Tutores 2120)} \label{sec:tutores}
Tutores\footnote{\url{https://www.urionlinejudge.com.br/judge/en/problems/view/2120}}

\section{Resultados} \label{sec:res}
% 
\subsection{Estruturas de Dados}
\label{sec:estruturas}

\subsubsection{Splay Tree}
\label{sec:splay}


\subsubsection{Hash table}
\label{sec:hash}

\usetikzlibrary{shapes.multipart,positioning,arrows,calc}
\tikzset{
  listnode/.style={
    rectangle split,rectangle split parts=2,draw,rectangle split horizontal,fill=blue!20
  },
  startnode/.style={
    draw,minimum width=1.5cm,minimum height=.75cm
  }
}

\begin{figure}
  \centering
  \begin{tikzpicture}[scale=.2, >=stealth]
    \node[startnode] (t0) {T[0]};
    \node[startnode,below=0pt of t0] (t1) {T[1]};
    \node[startnode,below=0pt of t1] (t2) {T[2]};
    \node[listnode,right=of t0] (3) {3};
    \node[listnode,right=of t1] (1) {1};
    \node[listnode,right=.5cm of 1] (7) {7};
    \node[listnode,right=of t2] (2) {2};
    \node[listnode,right=.5cm of 2] (5) {5};
    \node[listnode,right=.5cm of 5] (8) {8};
    \node[right=.5cm of 3] (3x) {$\varnothing$};
    \node[right=.5cm of 7] (7x) {$\varnothing$};
    \node[right=.5cm of 8] (8x) {$\varnothing$};
    \draw[*->] let \p1 = (3.two), \p2 = (3.center) in (\x1,\y2) -- (3x);
    \draw[*->] let \p1 = (1.two), \p2 = (1.center) in (\x1,\y2) -- (7);
    \draw[*->] let \p1 = (7.two), \p2 = (7.center) in (\x1,\y2) -- (7x);
    \draw[*->] let \p1 = (2.two), \p2 = (2.center) in (\x1,\y2) -- (5);
    \draw[*->] let \p1 = (5.two), \p2 = (5.center) in (\x1,\y2) -- (8);
    \draw[*->] let \p1 = (8.two), \p2 = (8.center) in (\x1,\y2) -- (8x);
    \draw[->] (t0) edge (3) (t1) edge (1) (t2) edge (2);
  \end{tikzpicture}

  \caption{Representação da tabela \textit{hash} com três \textit{buckets}
    implementados utilizando listas ligadas.}
  \label{fig:hash}
\end{figure}
\subsection{Algoritmo} \label{sec:algo}
\begin{algorithmic}[0]
  \Procedure{Nome}{$mat[N]$}\Comment{Vetor de matrículas de tamanho $N$.}
  \State $keys[N] \gets 0$\Comment{Vetor de chaves com tamanho $N$.}
  \State $order \gets$ \Call{SplayTree}{ }
  \State $parent \gets$ \Call{HashTable}{N}
  \State $level \gets$ \Call{HashTable}{N}

  \Statex
  \For{$i \gets 0, N$}
  \State $keys[i] \gets mat[i]$

  \If{$order \neq \emptyset$}
  \State $upper \gets$ \Call{LowerBound}{$order, mat[i]$}
  \If{$upper$ not found}
  \State $w \gets$ \Call{Max}{$order$} \Comment{Maior elemento de $order$.}
  \State $l \gets$ \Call{LookUp}{$level, w+1$} \Comment{$l = level[w+1]$}
  \State \Call{Insert}{$parent, mat[i], w$} \Comment{$parent[mat[i]] = w$}
  \State \Call{Insert}{$level, mat[i], l$}
  \ElsIf{$upper = $ \Call{Min}{$order$}} \Comment{Menor elemento de $order$}
  \State $w \gets$ \Call{Min}{$order$}
  \State $l \gets$ \Call{LookUp}{$level, w+1$}
  \State \Call{Insert}{$parent, mat[i],w$}
  \State \Call{Insert}{$level, mat[i],l$}
  \Else
  \State $y \gets upper$
  \State $lower \gets$ \Call{Previous}{$order, y$} \Comment{Elemento anterior a
    $y$ em $order$}
  \State $z \gets lower$
  \If{\Call{LookUp}{$level, y$} $>$ \Call{LookUp}{$level,z$}}
  \State $l \gets$ \Call{LookUp}{$level, y$}
  \State \Call{Insert}{$parent, mat[i], y$}
  \State \Call{Insert}{$level, mat[i], l+1$}
  \Else
  \State $l \gets$ \Call{LookUp}{$level, z$}
  \State \Call{Insert}{$parent, mat[i], z$}
  \State \Call{Insert}{$level, mat[i], l+1$}
  \EndIf
  \EndIf
  \EndIf
  
  \State \Call{Insert}{$order, mat[i]$}
  \State $i \gets i + 1$
  
  \EndFor
  \EndProcedure
    % \Procedure{Euclid}{$a,b$}\Comment{The g.c.d. of a and b}
    % \State $r\gets a\bmod b$
    % \While{$r\not=0$}\Comment{We have the answer if r is 0}
    %    \State $a\gets b$
    %    \State $b\gets r$
    %    \State $r\gets a\bmod b$
    % \EndWhile\label{euclidendwhile}
    % \State \textbf{return} $b$\Comment{The gcd is b}
    % \EndProcedure
\end{algorithmic}


% \begin{algorithm}[H]
%   \SetAlgoLined
%   \SetKwData{Int}{int}
%   \SetKwFunction{htbl}{HashTable}\SetKwFunction{stree}{SplayTree}
%   \SetKwFunction{read}{Read}\SetKwFunction{empty}{EmptyTree}
  
%   \KwResult{how to write algorithm with \LaTeX2e }
%   \BlankLine
%   keys[100100]\;
%   order $\leftarrow$ \stree{}\;
%   parent $\leftarrow$ \htbl{100100}\;
%   level $\leftarrow$ \htbl{100100}\;
%   \BlankLine
%   n $\leftarrow$ \read{int}\;
%   % \For{$i \leftarrow 0$ \KwTo n}{
%   %   x $\leftarrow$ \read{int}\;
%   %   keys[i] $\leftarrow$ x\;




    
%   % }

  
%   % \While{not at end of this document}{
%   %   read current\;
%   %   \eIf{understand}{
%   %     go to next section\;
%   %     current section becomes this one\;
%   %   }{
%   %     go back to the beginning of current section\;
%   %   }
%   % }
%   \caption{How to write algorithms}
% \end{algorithm}
\subsection{Testes}\label{sec:testes}
\subsubsection{Teste 1}
\begin{lstlisting}[frame=single,caption=Instância de teste 1,label=test1,escapeinside={\%*}{*)},inputencoding=utf8]
  3     // %*Número de estudantes*)
  5 1 2 // %*Ordem de inserção dos estudantes*)
  1     // %*Quantas consultas serão realizadas*)
  2     // %*Tutor do estudante 2*)
\end{lstlisting}

\begin{figure}[htb]
  \centering
  % dot -Gdpi=300 -Tpng test1.dot > test1.png
  \includegraphics[width=0.5\linewidth]{test1.png}
  \caption{Inserção}
  \label{fig:test1}
\end{figure}

\subsubsection{Teste 2}
\begin{figure}[htb]
  \centering
  % dot -Gdpi=300 -Tpng test2.dot > test2.png
  \includegraphics[width=\linewidth]{test2.png}
  \caption{Inserção}
  \label{fig:test1}
\end{figure}
\section{Conclusões}
\label{sec:conc}
Obtivemos um algoritmo capaz de realizar consultas aos tutores do colégio de
complexidade O($n \lg n$), superando o método anterior na ordem de O($n^2$).
Por este resultado resultado não há mais a necessidade de armazenar a ABB para
realizar consultas, reduzindo custos de armazenamento para a escola.
% \section{General Information}

% All full papers and posters (short papers) submitted to some SBC conference,
% including any supporting documents, should be written in English or in
% Portuguese. The format paper should be A4 with single column, 3.5 cm for upper
% margin, 2.5 cm for bottom margin and 3.0 cm for lateral margins, without
% headers or footers. The main font must be Times, 12 point nominal size, with 6
% points of space before each paragraph. Page numbers must be suppressed.

% Full papers must respect the page limits defined by the conference.
% Conferences that publish just abstracts ask for \textbf{one}-page texts.

% \section{First Page} \label{sec:firstpage}

% The first page must display the paper title, the name and address of the
% authors, the abstract in English and ``resumo'' in Portuguese (``resumos'' are
% required only for papers written in Portuguese). The title must be centered
% over the whole page, in 16 point boldface font and with 12 points of space
% before itself. Author names must be centered in 12 point font, bold, all of
% them disposed in the same line, separated by commas and with 12 points of
% space after the title. Addresses must be centered in 12 point font, also with
% 12 points of space after the authors' names. E-mail addresses should be
% written using font Courier New, 10 point nominal size, with 6 points of space
% before and 6 points of space after.

% The abstract and ``resumo'' (if is the case) must be in 12 point Times font,
% indented 0.8cm on both sides. The word \textbf{Abstract} and \textbf{Resumo},
% should be written in boldface and must precede the text.

% \section{CD-ROMs and Printed Proceedings}

% In some conferences, the papers are published on CD-ROM while only the
% abstract is published in the printed Proceedings. In this case, authors are
% invited to prepare two final versions of the paper. One, complete, to be
% published on the CD and the other, containing only the first page, with
% abstract and ``resumo'' (for papers in Portuguese).

% \section{Sections and Paragraphs}

% Section titles must be in boldface, 13pt, flush left. There should be an extra
% 12 pt of space before each title. Section numbering is optional. The first
% paragraph of each section should not be indented, while the first lines of
% subsequent paragraphs should be indented by 1.27 cm.

% \subsection{Subsections}

% The subsection titles must be in boldface, 12pt, flush left.

% \section{Figures and Captions}\label{sec:figs}


% Figure and table captions should be centered if less than one line
% (Figure~\ref{fig:exampleFig1}), otherwise justified and indented by 0.8cm on
% both margins, as shown in Figure~\ref{fig:exampleFig2}. The caption font must
% be Helvetica, 10 point, boldface, with 6 points of space before and after each
% caption.

% \begin{figure}[ht]
% \centering
% \includegraphics[width=.5\textwidth]{fig1.jpg}
% \caption{A typical figure}
% \label{fig:exampleFig1}
% \end{figure}

% \begin{figure}[ht]
% \centering
% \includegraphics[width=.3\textwidth]{fig2.jpg}
% \caption{This figure is an example of a figure caption taking more than one
%   line and justified considering margins mentioned in Section~\ref{sec:figs}.}
% \label{fig:exampleFig2}
% \end{figure}

% In tables, try to avoid the use of colored or shaded backgrounds, and avoid
% thick, doubled, or unnecessary framing lines. When reporting empirical data,
% do not use more decimal digits than warranted by their precision and
% reproducibility. Table caption must be placed before the table (see Table 1)
% and the font used must also be Helvetica, 10 point, boldface, with 6 points of
% space before and after each caption.

% \begin{table}[ht]
% \centering
% \caption{Variables to be considered on the evaluation of interaction
%   techniques}
% \label{tab:exTable1}
% \includegraphics[width=.7\textwidth]{table.jpg}
% \end{table}

% \section{Images}

% All images and illustrations should be in black-and-white, or gray tones,
% excepting for the papers that will be electronically available (on CD-ROMs,
% internet, etc.). The image resolution on paper should be about 600 dpi for
% black-and-white images, and 150-300 dpi for grayscale images.  Do not include
% images with excessive resolution, as they may take hours to print, without any
% visible difference in the result. 

% \section{References}

% Bibliographic references must be unambiguous and uniform.  We recommend giving
% the author names references in brackets, e.g. \cite{knuth:84},
% \cite{boulic:91}, and \cite{smith:99}.

% The references must be listed using 12 point font size, with 6 points of space
% before each reference. The first line of each reference should not be
% indented, while the subsequent should be indented by 0.5 cm.

\bibliographystyle{sbc}
\bibliography{sbc-template}

\end{document}
